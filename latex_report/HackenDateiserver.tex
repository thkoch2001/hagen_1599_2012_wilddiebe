\subsection{Dateiserver hacken}
\label{sec:dateiserverhacken}


\subsubsection{Zugriff auf den Dateiserver}

Um an die Rezepte zu kommen, versuchen wir über den \emph{Windows Explorer} auf den Dateiserver zuzugreifen.
Dazu geben wir den Pfad \textbackslash\textbackslash172.16.14.40 ein. Daraufhin werden insbesondere die Freigaben \path{public} und \path{Geheimrezepte} aufgelistet. Unter der Freigabe \path{public} ist das erste Rezept in verschlüsselter Form frei zugänglich\footnote{Die Entschlüsselung
  des Rezepts wird in \cref{RezeptEinsEntschluesseln} gezeigt}. Die Freigabe \path{Geheimrezepte} ist aber geschützt, so das wir darauf nicht zugreifen können. Da es sehr unwahrscheinlich ist, das richtige Kennwort zu erraten, werden wir auf Hacking-Tools zugreifen.

\subsubsection{Metasploit}

Das \Metasploit{}-Projekt ist ein freies Open-Source-Projekt zur
Computersicherheit, das Informationen über Sicherheitslücken bietet und bei
Penetrationstests sowie der Entwicklung von IDS-Signaturen(? @TODO) eingesetzt
werden kann. Das bekannteste Teilprojekt ist das \emph{Metasploit Framework}, ein
Werkzeug zur Entwicklung und Ausführung von Exploits gegen verteilte
Zielrechner. Andere wichtige Teilprojekte sind das Shellcode-Archiv und
Forschung im Bereich der IT-Sicherheit.

Für den Zugriff auf das \emph{Metasploit Framework} gibt es verschiedene Benutzeroberflächen. 
Eine davon ist \Armitage{}. Dabei handelt es sich um eine grafische Oberfläche, die eine einfache 
und intuitive Bedienung ermöglicht. Für unsere Hacking-Versuche wählen wir daher \Armitage.

Nach dem Programmstart müssen wir zuerst den anzugreifenden Computer bestimmen. Dazu wählen wir im Menü \emph{Hosts} 
den Eintrag \emph{\glqq{}Add Hosts\ldots{}\grqq{}} und fügen die IP-Adresse 172.16.14.40 hinzu.

Bevor wir mit der Schwachstellen-Analyse beginnen können, muss \Armitage{} Informationen über den anzugreifenden Rechner erhalten. Das erreichen wir, indem wir über das Kontextmenü des Hosts einen \emph{Scan} starten. \Armitage{} bekommt so eine Liste der offenen Ports und der verfügbaren Dienste und kann mit der Schwachstellen-Analyse beginnen.

Hierzu wählen wir im Menü \emph{Attacks} den Befehl \emph{\glqq{}Find Attacks\grqq{}}. \Armitage{} zeigt daraufhin \emph{mögliche} Angriffspunkte (sog. \emph{Exploits}) in den Kategorien \texttt{dcerpc}, \texttt{ids}, \texttt{oracle}, \texttt{samba} und \texttt{smb} an. Ob das System tatsächlich über diese Exploits angreifbar ist, prüfen wir mit der Funktion \emph{\glqq{}check exploits\ldots\grqq{}}. Nur für den Exploit \texttt{windows/smb/ms08\_067\_netapi} wird eine Verwundbarkeit bestätigt. (siehe \bildref{DateiserverHacken1}).

\bilddatei{DateiserverHacken1}{Check Exploits}{0.7}

Über das Kontextmenü rufen wir diesen Exploit auf. Der erfolgreiche Angriff ist sofort grafisch zu erkennen, weil das Host 
jetzt rot dargestellt wir und mit Blitzen versehen ist. Jetzt können wir verschiedene Funktionen auf dem Host durchführen.
Um das zweite Rezept herunter zu laden gehen wir im Kontextmenü über \emph{Meterpreter 1} und \emph{Explore} zu \emph{\glqq{}Browse files\grqq{}}. Daraufhin erscheint eine Registerkarte, mit den Dateien und Verzeichnissen des Dateiservers. Jetzt können zum Verzeichnis \path{C:\Geheimrezepte} navigieren und anschließend über das Kontextmenü das verschlüsselte Rezept\footnote{Die Entschlüsselung des Rezepts wird in \cref{RezeptZweiEntschluesseln} gezeigt} herunterladen (siehe \bildref{DateiserverHacken3}).

\bilddateisloppy{DateiserverHacken3}{Datei herunterladen}{0.7}

Aus den offenen Ports des Rechners \path{datei.mayerbrot.local} kann man
schließen, dass es sich um einen Windows Server handelt. Es wurde mit
\\172.16.14.40 versucht, sich die Freigaben anzuzeigen. Es war zu diesem
Zeitpunkt nur public zu sehen. Auf das public Verzeichnis kann von einem Linux
oder Free-BSD Rechner einfach zugegriffen werden. Dazu wird die Freigabe
\glqq{}gemountet\grqq{} und die Dateien können mit regulären UNIX Befehlen angesehen und
kopiert werden:

\subsubsection{Zugriff auf den Dateiserver}

Aus den offenen Ports des Rechners \path{datei.mayerbrot.local} kann man
schließen, dass es sich um einen Windows Server handelt. Es wurde mit
\\172.16.14.40 versucht, sich die Freigaben anzuzeigen. Es war zu diesem
Zeitpunkt nur public zu sehen. Auf das public Verzeichnis kann von einem Linux
oder Free-BSD Rechner einfach zugegriffen werden. Dazu wird die Freigabe
\glqq{}gemountet\grqq{} und die Dateien können mit regulären UNIX Befehlen angesehen und
kopiert werden:

\begin{lstlisting}
\% mkdir /mnt/cifs
\% sudo mount -t cifs //172.16.14.40/public /mnt/cifs
\% cd /mnt/cifs
\% ls
Rezept1NR.txt  testdatei.txt
\% cp -v Rezept1NR.txt /home/xxx
`Rezept1NR.txt' -> `/home/xxx/Rezept1NR.txt'
\end{lstlisting}

\subsubsection{Metasploit}


\begin{lstlisting}

\end{lstlisting}
