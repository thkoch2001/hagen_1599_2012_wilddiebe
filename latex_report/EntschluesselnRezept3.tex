\subsection{Rezept 3 entschlüsseln}
\label{RezeptDreiEntschluesseln}

Das verschlüsselte dritte Rezept:

{\tiny
  \verbatiminput{verschluesselte_rezepte/rezept3.txt}
}

Die verschlüsselte Datei wurde zunächste einer Häufigkeitsanalyse in
JCrypTool\footnote{Java Lehrsoftware für Kryptologie, erhältlich unter
  \url{http://www.cryptool.org}} unterzogen. Als Einstellungen wurden
\glqq{}Monoalphabetisch\grqq{} sowie als Referenzverteilung \glqq{}DEUTSCH Descartes\grqq{}
ausgewählt. Als ursprüngliches Alphabet des Klartextes wurde \glqq{}Printable ASCII\grqq{} 
genommen. Die Häufigkeitsanalyse zeigt \cref{fig:rezept3verteilung1}.

\bilddatei{rezept3verteilung1}{}{}

Durch einfaches Ziehen mit der Maus konnte man die tatsächliche Verteilung
(schwarz) mit der Referenzverteilung (weiß) in Übereinstimmung bringen. Dabei
wurde der Graph um 47 Stellen verschoben (\cref{fig:rezept3verteilung2}).

\bilddatei{rezept3verteilung2}{}{}

Mit Hilfe der Häufigkeitsverteilung kann man genau sehen, dass das kleine \glqq{}e\grqq{} 
in die Zahl \glqq{}6\grqq{} verschlüsselt wurde. Leider konnte mit den in JCrypTool
vorhandenen Alphabeten keine Entschlüsselung gelingen. So wurden zunächst
manuell mit Hilfe des Editors und Suchen und Ersetzen die Geheimschriftzeichen
durch die Klartextzeichen ersetzt. Es stellte sich dabei heraus, dass die
Umlaute und die Leerstellen nicht verschlüsselt wurden. Mit Hilfe eines eigens
angepaßten Alphabets konnte dann anschließend die Entschlüsselung auch mit dem
JCrypTool automatisiert werden. Das angepaßte Alphabet ist Printable ASCII ohne
Leerstellen und Umlaute (\cref{fig:rezept3alphabet}).

\bilddatei{rezept3alphabet}{}{}

Im Menü von JCrypTool unter Fenster $\rightarrow$ Benutzervorgaben $\rightarrow$
Kryptographie $\rightarrow$ Alphabete wurde das benutzerdefinierte Alphabet
hinzugefügt. Weiterhin wurde der Haken entfernt bei der Option \glqq{}Filtern von
ungültigen Zeichen\grqq{}.  Mit Hilfe des angepaßten Alphabets und der Wahl von \glqq{}O\grqq{}
als Schlüssel kann das Rezept nun ganz einfach entschlüsselt werden.  Der
angewandte Algorithmus ist Caesar-Chiffre mit einer Verschiebung um 47 Zeichen.

Das entschlüsselte Rezept lautet dann:

\begin{quote}
 Brötchen

Mehl und Salz in einer Schüssel verrühren. Die Hefe im lauwarmen Wasser auflösen und den Zucker unterrühren.Eine Vertiefung in das Mehl drücken und das Olivennöl auf den Rand geben. Die Hefe-Zucker-Wasser-Masse in die Mulde giessen und alles mit dem Rührgerät kneten, bis ein gleichmässiger Teig entstanden ist. Der Teig kann ruhig etwas klebrig sein. Zur Weiterverarbeitung von aussen mit Mehl bestäuben. Den Teig ca. 45 Minuten gehen lassen. Den Teig noch einmal gut durchkneten und 8 - 10 Brötchen formen. Die Brötchen im Backofen bei 50 Grad ca. 10 Minuten gehen lassen. Eine feuerfeste Schüssel mit Wasser in den Backofen stellen, die Brötchen mit Wasser bestreichen und den Ofen auf 230 Grad heizen.Die Brötchen einschieben und backen, bis sie anfangen braun zu werden. Wer die Kruste gerne etwas weicher hat, kann die Brötchen während des Backens und vor dem Abkühlen erneut mit Wasser bestreichen. Bei Vollkornbrötchen kann man z. B. 340 g Weizenmehl (Typ 1050) und 160 g Roggenmehl verwenden und die Brötchen nach der ersten Ruhezeit mit Körnern mischen und nach Belieben auch bestreuen.

Zutaten: 500 g Mehl, 1 Würfel Hefe, 340 ml lauwarmes Wasser, 1 TL Zucker, 2 TL Salz, 3 TL Olivenöl 
\end{quote}
