\subsection{Rezept 2 entschlüsseln}
\label{RezeptZweiEntschluesseln}

{\tiny
\begin{lstlisting}[caption={Das verschlüsselte 2. Rezept: Rezept2NR.txt},
 linewidth=\textwidth,
 breaklines, breakatwhitespace=false]
HrSbkesnuz1eWbsgteeamhWbezldsiDeSgeraeuWgmneSnmmunnant
solsrivtieuMsusrnk3eieteean0vzBribAFhkau2gh15reailsron
n0eeeinceraa1zhsreboeonlnednnaudBciet0himareesbekqlnsn
WleicemareknzinanasegenisznaneDnetdsgnlnsarurieadc3teb
imeptreleDateSreaVkrhuaSfaMnneeeenftenteugabatomeacedS
dnZmtmauhnsnBcea2GCoiesraiMnnanuroembhlsZtn5ogele17get
WisuegmkeWe1Vlrwem5gnnmnn2ene2uhzngohcWle0l1leeWeietin
euereLimndnheni0lseserieadcullsDenstonesedckreesesarbs
eesueaebnSuegRgmhdaseteshbek1udeZeteaghasiSedaeidolnen
dazctuernnneettdihaettKefgbbenntnbimrprrelsDnkfu2rdren
aodn0uecesernealnsnae0RemTp51Suegznet30atas7gloezelSnb
ukr5Lsm5Bwi5gbekensga0misasdDnmneeewmmssgneeaiskdseuhs
WasNdteaioedsvcgtnirresenuglmdlhtkIgemlseneduueeegaeof
ahdt5tkdmneetgny0aier0ls5kiholegage0gtsS0ss
\end{lstlisting}
}

Da lediglich wieder nur der chiffrierte Text bekannt war, wurde Ciphertext-Only
gewählt.

Die Häufigkeitsanalyse ergab, dass \glqq{}e\grqq{} wieder am häufigsten auftritt. Das
Dechiffrieren des Textbeginns mit der Caesar Chiffre ergibt ein sinnloses
\glqq{}EOPYHBPKRW1\grqq{}, rot-13 ergibt \glqq{}UEFOXRFAHM1\grqq{}.

\begin{table}[H]
\centering
\begin{tabular}{ccccc}
\toprule
\begin{tabular}{cc}
13 & 0\\
8 & 1\\
5 & 2\\
3 & 3\\
9 & 5\\
2 & 7\\
1 & A\\
4 & B\\
1 & C\\
6 & D\\
\end{tabular}
&
\begin{tabular}{cc}
1 & F\\
1 & G\\
1 & H\\
1 & I\\
1 & K\\
2 & L\\
3 & M\\
1 & N\\
2 & R\\
11 & S\\
\end{tabular}
&
\begin{tabular}{cc}
1 & T\\
2 & V\\
9 & W\\
3 & Z\\
50 & a\\
17 & b\\
12 & c\\
26 & d\\
134 & e\\
5 & f\\
\end{tabular}
&
\begin{tabular}{cc}
29 & g\\
19 & h\\
33 & i\\
16 & k\\
27 & l\\
26 & m\\
75 & n\\
16 & o\\
3 & p\\
1 & q\\
\end{tabular}
&
\begin{tabular}{cc}
34 & r\\
52 & s\\
34 & t\\
28 & u\\
3 & v\\
3 & w\\
1 & y\\
10 & z\\
&\\
&\\
\end{tabular}\\
\bottomrule
\end{tabular}
\caption{Häufigkeitsanalyse}
\label{tab:Häufigkeitsanalyse2}
\end{table}

\subsubsection*{Rail Fence Cipher}

Diese Chiffre nennt sich Rail Fence Cipher, aufgrund der Art wie der
Chiffriervorgang abläuft. Der Klartext wird diagonal nach rechts unten auf einen
imaginären Zaun geschrieben. Wenn das untere Ende erreicht wurde, wird diagonal
nach rechts oben bis zum oberen Ende geschrieben, usw. bis der gesamte Klartext
geschrieben ist.

Mit einer Tiefe von zwei entsteht aus dem Beginn von Rezept 2:

\begin{lstlisting}
H r S b k e s n u z 1 e W
 i e t i n e u e r e L i m
\end{lstlisting}

Mit einer Tiefe von drei entsteht

\begin{lstlisting}
H   r   S   b   k   e   s
 z n a n e D n e t d s g
  a   e   0   R   e   m
\end{lstlisting}
Und mit einer Tiefe von vier schließlich lässt sich ein Rezept erkennen:
\begin{lstlisting}
H     r     S     b     k
 e   b o   e o   n l   n
  i e   t i   n e   u e
   d     D     n     m
\end{lstlisting}

Das Programm aus Abbildung ~\ref{fig:rafeci} wurde geschrieben und benutzt um den Text zu
entschlüsseln.

\begin{figure}[p]
\lstinputlisting[language=C,basicstyle=\scriptsize]{listings/rafeci.c}
\caption{rafeci.c}
\label{fig:rafeci}
\end{figure}

Das entschlüsselte Rezept lautet:

{\tiny
\lstinputlisting[inputencoding=latin1]{entschluesselte_rezepte/Rezept2NRDatE.txt}
}
