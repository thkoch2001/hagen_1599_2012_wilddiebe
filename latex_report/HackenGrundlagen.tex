\section{Hacken}
\subsection{Grundlagen}

Es lassen sich zwei grundlegende Formen des Hackens unterscheiden -- Social
Engineering und direktes, klassisch-technisches Hacken. Die beiden Formen können
aber natürlich auch in Kombination angewendet werden.

\subsubsection{Social Engineering}

Social Engineering bedeutet, dass der Angreifer durch eine erfundene Geschichte
einen berechtigten Nutzer dazu bringt, etwas zu tun, das dem Angreifer hilft die
Sicherheitsbarrieren des Systems zu überwinden \cite[S.~77]{Pip00}. Am
gefährlichsten ist es, wenn Administratoren einer Social-Engineering-Attacke
erliegen \cite[S.~94]{Rae01}. Die Kontaktaufnahme kann persönlich, über E-Mail
oder über Telefon erfolgen \cite[S.~119]{Sta95}.

Als Beispiel führen Köhntopp et al. an: Der Angreifer ruft einen Mitarbeiter an
und gibt sich dabei als Kollege aus einer anderen Abteilung oder als
Systemadministrator aus. Dann bittet er seinen Gesprächspartner um ein Passwort
oder um sonstige Angaben, angeblich damit er noch schnell eine dringende Arbeit
erledigen oder einen Systemfehler beheben kann \cite[S.~15]{KSG98}. So gelangt
der Angreifer direkt an die nötigen Zugriffskennungen oder er nutzt die
erhaltenen Informationen für seinen nächsten Schritt, etwa um sich das Vertrauen
des nachfolgenden Opfers zu erschleichen und von diesem weitergehendere
Auskünfte einzuholen \cite[S.~260]{ScB01}.

Nimmt ein Angreifer per E-Mail Kontakt auf, kann er Janowicz zufolge \zB{}
mittels E-Mail-Spoofing Nachrichten mit einer beliebigen Absenderadresse
versehen. Falls der Betrüger in einer derartigen Mail einen anderen Antwortpfad
spezifiziert hat und der Empfänger auf den Antwort-Button klickt, erreicht seine
Mitteilung zudem nicht den vorgeblichen Absender, sondern geht an die vom
Angreifer gewählte Adresse. Selbst ganz ohne technische Tricks kann jemand
versuchen, den Empfänger einer E-Mail über deren Herkunft zu täuschen, indem er
bei einem Freemail-Anbieter einen Account mit einer irreführenden Adresse
einrichtet, die den Namen einer vertrauenswürdigen Person oder Organisation
enthält \cite[S.~109--110, 123]{Jan02}.

\subsubsection{Direktes, klassisch-technisches Hacken}

Unter direktem, klassisch-technischem Hacken werden \va{} netzwerkbasierte
Angriffe gefasst, die von Personen zur Erlangung von Zugriffsrechten
durchgeführt werden und dabei als Ansatzpunkte die Gefährdungen nutzen, die sich
aus der Internetanbindung des Unternehmens bzw. der Verwendung eines Intranets
ergeben \cite[S. 363]{Sta01}. Der Angreifer agiert entweder über das Internet
oder befindet sich von vornherein im internen Netzwerk.

Üblicherweise sammelt ein Hacker, wie Fuhrberg et al. darlegen, zunächst
Informationen über das System und sucht nach Schwachstellen. Beispielsweise kann
der Hacker versuchen, mit einem Portscan die offenen Ports des Zielcomputers zu
erkennen und dann die jeweils dahinterliegenden Programme bzw. Programmversionen
festzustellen. Viele Dienste verraten standardmäßig Informationen über sich oder
das zugrundeliegende Betriebssystem, etwa in ihren Login- oder Fehlermeldungen
\cite[S.~58-59]{FHW01}. Manche interessante Angaben sind unmittelbar öffentlich
verfügbar, etwa auf der Website des Unternehmens. Eventuell ist auch der
Name-Server der Firma so eingerichtet, dass er seine Daten jedermann preisgibt
und dadurch Hinweise auf die Struktur des internen Netzes liefert
\cite[S.~222]{Pip00}.

Anschließend nutzt der Hacker ihm bekannte Schwachpunkte des jeweiligen Systems
aus \cite[S.~97]{Poh01}. Die Schwachstellen, die Hacker ausnutzen können,
entstehen durch Konzeptions-, Programmier- oder Konfigurationsfehler
\cite[S. 48]{FHW01}. Das Sammeln von Informationen und Ausnutzen von
Schwachstellen wird ggf. mehrmals wiederholt, bis der Eindringling alle
Schutzmechanismen überwunden und sein endgültiges Ziel erreicht hat
\cite[S.~98--99]{Rae01}.

Der Hacker kann den gesamten Prozess manuell durchführen oder ihn mit Hilfe von
oftmals frei im Internet erhältlichen Tools teilweise oder ganz automatisieren
\cite[S. 52-53]{SSF02}. Auch eine sehr große Anzahl an sogenannten \glqq{}Script
Kiddies\grqq{}, technischen Laien, verwendet die automatisierten Tools und führt
damit gefährliche Angriffe durch \cite[S.~82]{CoM99}.

Cheswick/Matzer erläutern, dass nach einem erfolgreichen Einbruch der Hacker
wahrscheinlich versuchen wird, die elektronischen Spuren, die er hinterlassen
hat, zu verwischen und eine Hintertür zu installieren, damit er später leichter
zurückkehren kann. Vom eroberten Computer aus kann der Eindringling weitere
Server im Netzwerk angreifen und unter seine Kontrolle bringen
\cite[S.~182]{ChB99}.  Dabei nutzt er das transitive Vertrauen aus, \dH{} er
verwendet die erweiterten Zugriffsrechte, über die der bereits eingenommene
Rechner bei der nächsten Zielmaschine verfügt \cite[S.~14]{KSG98}.

Im schlimmsten Fall erreicht der Hacker volle Kontrolle über das System und kann
sämtliche Informationen einsehen, ändern oder löschen, die technisch gesehen
über das Netzwerk erreichbar bzw. manipulierbar sind. Der Angreifer ist \zB{}
auch in der Lage, die unterwanderten Server als Sprungbrett für Angriffe auf
andere Firmen im Internet zu verwenden, wodurch aus Sicht des Opfers das zuerst
infiltrierte Unternehmen als Urheber der Attacken erscheint
\cite[S.~364--365]{Sta01}. Oder der Hacker legt laut Stiefenhofer auf den
Rechnern Dateien mit illegalen Inhalten ab und bietet sie auf Kosten des
Unternehmens zum Download an. Das Bekanntwerden eines erfolgreichen Einbruchs
kann zu einem beachtlichen Image- und Vertrauensverlust führen
\cite[S.~38]{ScO97}.
 






