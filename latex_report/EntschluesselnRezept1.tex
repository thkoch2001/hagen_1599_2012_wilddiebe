\subsection{Rezept 1 entschlüsseln}
\label{RezeptEinsEntschluesseln}

{\tiny
\lstinputlisting[caption={Das verschlüsselte 1. Rezept: Rezept1NR.txt},
 linewidth=\textwidth,
 breaklines,
  breakatwhitespace=true,
  inputencoding=latin1,
  tabsize=20]{verschluesselte_rezepte/Rezept1NR.txt}
}

Die Häufigkeit der einzelnen Zeichen im verschlüsselten Rezept wurde mit einem
Onlinetool\footnote{\url{http://www.woerter-zaehlen.de/index.php}} gezählt und
ist in Tabelle \ref{tab:rezept1haeufigkeit} wiedergegeben.

Die Häufigkeitsanalyse ergab, dass \glqq{}e\grqq{} am häufigsten auftritt.
Es wurde als nächstes versucht, mittels der Caesar Chiffre das Wort
\glqq{}Znaqryfghgra\grqq{} zu analysieren. Eine Verschiebung um 3 nach links,
also mit dem Schlüssel 3, ergibt folgende Ansicht:

\begin{lstlisting}
ABCDEFGHIJKLMNOPQRSTUVWXYZ
DEFGHIJKLMNOPQRSTUVWXYZABC

ZNAQRYFGHGRA  -> WKXNOVCDEDOX
\end{lstlisting}

\begin{table}[H]
\centering
\begin{tabular}{ccccc}
\toprule
	\begin{tabular}{cc}
10 & " "\\
10 & " "\\
126 & " "\\
13 & ","\\
1 & "-"\\
8 & "."\\
10 & "0"\\
4 & "1"\\
3 & "2"\\
2 & "3"\\
3 & "4"\\
3 & "5"\\
	\end{tabular}
&
	\begin{tabular}{cc}
1 & "6"\\
1 & ":"\\
1 & "B"\\
5 & "F"\\
2 & "G"\\
1 & "H"\\
1 & "I"\\
2 & "J"\\
4 & "M"\\
3 & "N"\\
5 & "O"\\
3 & "Q"\\
	\end{tabular}
&
\begin{tabular}{cc}
4 & "R"\\
1 & "T"\\
2 & "U"\\
1 & "X"\\
1 & "Y"\\
10 & "Z"\\
58 & "a"\\
10 & "b"\\
1 & "c"\\
1 & "d"\\
38 & "e"\\
25 & "f"\\
	\end{tabular}
&
\begin{tabular}{cc}
44 & "g"\\
33 & "h"\\
4 & "i"\\
8 & "j"\\
5 & "m"\\
32 & "n"\\
5 & "o"\\
21 & "p"\\
21 & "q"\\
105 & "r"\\
13 & "s"\\
19 & "t"\\
	\end{tabular}

&

\begin{tabular}{cc}
20 & "u"\\
36 & "v"\\
8 & "x"\\
38 & "y"\\
18 & "z"\\
3 & "ˆ"\\
1 & "‰"\\
5 & "¸"\\
5 & "Å"\\
&\\
&\\
&\\
	\end{tabular}\\
\bottomrule
\end{tabular}
\caption{Häufigkeitsanalyse}
\label{tab:rezept1haeufigkeit}
\end{table}

\paragraph{Rot-13}

Als nächstes wurde mittels Rot-13 das Wort \glqq{}ZNAQRYFGHGRA\grqq{} analysiert.

Werden als Alphabet die Buchstaben A-Z gewählt und ein Schlüssel von 13, so wird
diese Chiffre auch als Rot-13 bezeichnet. Eine Verschiebung um 13 nach links,
also mit dem Schlüssel 13, ergibt folgende Ansicht:

\begin{lstlisting}
A B C D E F G H I J K L M
N O P Q R S T U V W X Y Z

ZNAQRYFGHGRA -> MANDELSTUTEN
\end{lstlisting}

Es wurde das Programm aus Abbildung ~\ref{fig:listingrot13} geschrieben und
benutzt um das Rezept 1 zu entschlüsseln.

\begin{figure}[tb]
\lstinputlisting[basicstyle=\scriptsize,language=C]{listings/rot13.c}
\caption{Quellcode rot13.c}
\label{fig:listingrot13}
\end{figure}

Das entschlüsselte Rezept lautet also vollständig:

{\tiny
\begin{quote}
Mandelstuten
Mehl, Zucker, 2 Eier, Bittermandelaroma, gehackte Mandeln und Butter in eine große Schüssel geben.
Die Hefe in der lauwarmen Milch vollständig auflösen und ebenfalls in die Schüssel zu den anderen Zutaten schütten. Alles gut verkneten und zu einer Kugel formen.
An einem warmen Ort zugedeckt 1-2 Stunden gehen lassen, bis sich das Volumen deutlich vergrößert hat.
Den Teig nochmals gut kneten und in eine gut gefettete große Brot Form füllen.
Die zwei restlichen Eier mit etwas lauwarmer Milch verquirlen und auf den Stuten streichen.
Auf mittlerer Schiene bei 200 Grad Umluft ca. 35 Minuten backen.

Zutaten: 1000g Weizenmehl, 400ml lauwarme Milch, 60g Zucker, 1 Würfel Hefe, 4 Eier, 50g weiche Butter, 3 Tropfen Bittermandelöl,
150g gehackte Mandeln, 4 EL lauwarme Milch
\end{quote}
}
